\documentclass[10pt,letterpaper,fleqn]{article}

\usepackage[utf8]{inputenc}
\usepackage[spanish,es-nodecimaldot]{babel}
\usepackage{amsmath}
\usepackage{amssymb}
\usepackage{multicol}
\usepackage{graphicx}
\usepackage{cancel}

\usepackage[dvipsnames]{xcolor}
\usepackage[most]{tcolorbox}

\usepackage{tabu}

\usepackage{pgfplots}
\pgfplotsset{width=10cm,compat=1.9}

\usepackage{mathtools}
\usepackage{tikz}
\usetikzlibrary{trees,positioning}

\usepackage[top=1in, bottom=1in, left=1in, right=1in]{geometry}


\begin{document}

\begin{titlepage}
    \centering

    {\scshape\LARGE Universidad Nacional Autónoma de México \par}

    \vspace{1cm}
    {\scshape\Large Facultad de ciencias\par}
    \vspace{1.5cm}

    \begin{center}
        \includegraphics[scale=.1]{../../assets/img/logo.png}
    \end{center}

    \vspace{.8 cm}

    {\LARGE Tarea  02 : \par}
    {\huge\bfseries Logica Computacional\par}

    \vspace{0.5cm}
    \large{\itshape{Alex Gerardo Fernández Aguilar}} \small{ -314338097 } \\ \vspace{0.3cm}
    
   
    \vfill

    Trabajo presentado como parte del curso de
    \textbf{Materia}
    impartido por el profesor \textbf{ Francisco Hernández Quiroz }. \par
    \vspace{0.1cm}
\end{titlepage}

    \begin{enumerate}
    
        %Ejercicio 1%
        \item \textbf{ Demuestra que si $f : \{V, F\}^n \rightarrow \{V, F\} (1 \leq n)$, entonces $f$ se puede definir en términos de
        alguna de las siguientes opciones (elige la que prefieras):}
        \\\textbf{usando $\neg  $  y $\vee $ }
        \begin{equation*}
            \begin{split}
                & \neg \,\neg \alpha \Leftrightarrow \alpha
                \\& var \Rightarrow var
                \\& (\alpha \vee \beta) \Rightarrow (\alpha \vee \beta)
                \\& (\alpha \wedge \beta) \Rightarrow \neg(\neg \alpha \vee \neg \beta)
                \\& (\alpha \rightarrow \beta) \Leftrightarrow (\neg \alpha \vee \beta)
                \\& (\alpha \leftrightarrow \beta) \Leftrightarrow \neg(\neg(\neg \alpha \vee \beta )\vee \neg(\neg \beta \vee   \alpha))
            \end{split}
        \end{equation*}
		
        %Ejercicio 2%
        \item \textbf{Transforma las siguientes fórmulas a formas normales conjuntivas:}
        \begin{enumerate}
            %Ejercicio 2.a%
            \item \textbf{ $(p \Rightarrow q) \wedge (r \vee \neg q) $}
            \begin{equation*}
                \begin{split}
                    &   (p \Rightarrow q) \wedge (r \vee \neg q)
                    \\& (\neg p \vee  q) \wedge (r \vee \neg q)
                    \\& (\neg p \wedge  (r \vee \neg q)  ) \vee (q \wedge  (r \vee \neg q)  )
                    \\& ((\neg p \wedge r) \vee (\neg p \wedge \neg q)) \vee ((q \wedge r) \vee (q \wedge \neg q))
                    \\& (\neg p \wedge r) \vee (\neg p \wedge \neg q) \vee  (q \wedge r) \vee (q \wedge \neg q)
                \end{split}
            \end{equation*}
            
            %Ejercicio 2.b%
            \item \textbf{ $ (p \Leftrightarrow r) \vee \neg q $}
            \begin{equation*}
                \begin{split}
                    &   (p \Leftrightarrow r) \vee \neg q
                    \\& ( (p \Rightarrow r) \wedge (r \Rightarrow p) )  \vee \neg q
                    \\& ( (\neg p \vee r) \wedge (\neg r \vee p) )  \vee \neg q
                    \\& ( (\neg p \wedge (\neg r \vee p) ) \vee ( (r \wedge (\neg r \vee p) )  \vee \neg q
                    \\& ( (\neg p \wedge \neg r) \vee (\neg p \wedge p) )  \vee( (r \wedge \neg r) \vee (r \wedge p) ) \vee \neg q
                    \\& (\neg p \wedge \neg r) \vee (\neg p \wedge p) \vee (r \wedge \neg r) \vee (r \wedge p)\vee \neg q
                \end{split}
            \end{equation*}
            
            %Ejercicio 2.c%
            \item \textbf{ $\neg(p \vee q) \Leftrightarrow (q \vee \neg r) $}
            \begin{equation*}
                \begin{split}
                    &   (p \Rightarrow q) \wedge (r \vee \neg q)
                    \\& (\neg p \vee  q) \wedge (r \vee \neg q)
                    \\& (\neg p \wedge  (r \vee \neg q)  ) \vee (q \wedge  (r \vee \neg q)  )
                    \\& ((\neg p \wedge r) \vee (\neg p \wedge \neg q)) \vee ((q \wedge r) \vee (q \wedge \neg q))
                    \\& (\neg p \wedge r) \vee (\neg p \wedge \neg q) \vee  (q \wedge r) \vee (q \wedge \neg q)
                \end{split}
            \end{equation*}
            
            %Ejercicio 2.d%
            \item \textbf{ $p \Leftrightarrow ( q \Leftrightarrow ( r \Leftrightarrow p_1 ) ) $}
            \begin{equation*}
                \begin{split}
                    &   p \Leftrightarrow ( q \Leftrightarrow ( r \Leftrightarrow p_1 ) ) 
                    \\&  ( p \Rightarrow [ ( q \Rightarrow [ (r \Rightarrow p_1) \wedge (p_1 \Rightarrow r) ] ) \wedge ( [ ( r \Rightarrow p_1) \wedge (p_1 \Rightarrow r) ] \Rightarrow q ) ] ) 
                    \\& \wedge ( [ ( q \Rightarrow [ (r \Rightarrow p_1) \wedge (p_1 \Rightarrow r) ] ) \wedge ( [ ( r \Rightarrow p_1) \wedge (p_1 \Rightarrow r) ] \Rightarrow q ) ] \Rightarrow p )
                    \\&
                    \\& (\neg p \vee [ ( \neg q \vee [ (\neg r \vee p_1) \wedge (\neg p_1 \vee  r)])\wedge([(r \wedge \neg p_1) \vee (p_1 \wedge \neg r) ] \vee q ) ] ) 
                    \\& \wedge ( [ ( q \wedge [ (r \wedge \neg p_1) \vee (p_1 \wedge \neg r) ] ) \vee  ( [ (\neg r \vee p_1) \wedge (\neg p_1 \vee r) ] \wedge \neg q ) ] \Rightarrow p )
                    \\&
                    \\& (\neg p \wedge \neg (\neg q \wedge \neg (\neg r \wedge \neg p_1) \vee (\neg r \wedge r) \vee (p_1 \wedge \neg p_1) \vee (p_1 \wedge r)) 
                    \\& \vee (\neg q \wedge q) \vee ((\neg r \wedge \neg p_1) \vee (\neg r \wedge r) 
                    \\& \vee (p_1 \wedge \neg p_1) \vee (p_1 \wedge r) \wedge \neg (\neg r \wedge \neg p_1) \vee (\neg r \wedge r) \vee (p_1 \wedge \neg p_1) \vee (p_1 \wedge r)) \vee ((\neg r \wedge \neg p_1) 
                    \\& \vee (\neg r \wedge r) \vee (p_1 \wedge \neg p_1) \vee (p_1 \wedge r) \wedge q)) \vee (\neg p \wedge p) \vee ((\neg q \wedge \neg (\neg r \wedge \neg p_1) 
                    \\& \vee (\neg r \wedge r) \vee (p_1 \wedge \neg p_1) \vee (p_1 \wedge r)) \vee (\neg q \wedge q) \vee ((\neg r \wedge \neg p_1) \vee (\neg r \wedge r) 
                    \\& \vee (p_1 \wedge \neg p_1) \vee (p_1 \wedge r) \wedge \neg (\neg r \wedge \neg p_1) \vee (\neg r \wedge r) \vee (p_1 \wedge \neg p_1) \vee (p_1 \wedge r)) 
                    \\& \vee ((\neg r \wedge \neg p_1) \vee (\neg r \wedge r) \vee (p_1 \wedge \neg p_1) \vee (p_1 \wedge r) \wedge q) \wedge \neg (\neg q \wedge \neg (\neg r \wedge \neg p_1) 
                    \\& \vee (\neg r \wedge r) \vee (p_1 \wedge \neg p_1) \vee (p_1 \wedge r)) \vee (\neg q \wedge q) \vee ((\neg r \wedge \neg p_1) \vee (\neg r \wedge r) \vee (p_1 \wedge \neg p_1) 
                    \\& \vee (p_1 \wedge r) \wedge \neg (\neg r \wedge \neg p_1) \vee (\neg r \wedge r) \vee (p_1 \wedge \neg p_1) \vee (p_1 \wedge r)) \vee ((\neg r \wedge \neg p_1) 
                    \\& \vee (\neg r \wedge r) \vee (p_1 \wedge \neg p_1) \vee (p_1 \wedge r) \wedge q)) \vee ((\neg q \wedge \neg (\neg r \wedge \neg p_1) \vee (\neg r \wedge r) 
                    \\& \vee (p_1 \wedge \neg p_1) \vee (p_1 \wedge r)) \vee (\neg q \wedge q) \vee ((\neg r \wedge \neg p_1) \vee (\neg r \wedge r) \vee (p_1 \wedge \neg p_1) 
                    \\& \vee (p_1 \wedge r) \wedge \neg (\neg r \wedge \neg p_1) \vee (\neg r \wedge r) \vee (p_1 \wedge \neg p_1) \vee (p_1 \wedge r)) \vee ((\neg r \wedge \neg p_1) 
                    \\& \vee (\neg r \wedge r) \vee (p_1 \wedge \neg p_1) \vee (p_1 \wedge r) \wedge q) \wedge p)
                    \\&
                \end{split}
            \end{equation*}
            
        \end{enumerate}
        
        %Ejercicio 3%
        \item \textbf{  Demuestra los siguientes teoremas de deducción natural }
        \begin{enumerate}
            %ejercicio3.a
            \item \textbf{$\vdash_N (p \wedge q \Rightarrow r) \Leftrightarrow (p \Rightarrow (q \Rightarrow r));$}
            \\\begin{tabular}{l l r}
                   1  & $\vdash_N (p \wedge q \Rightarrow r) \Leftrightarrow (p \Rightarrow (q \Rightarrow r)) $ & (premisa)
                \\ 2  & $ [(p \wedge q \Rightarrow r) \Rightarrow (p \Rightarrow (q \Rightarrow r))] , [(p \Rightarrow (q \Rightarrow r)) \Rightarrow(p \wedge q \Rightarrow r)] \vdash_N $ & (premisa) 
                \\3  & $ (p \wedge r ) \vdash (q \wedge r) $ &(Eliminacion $\rightarrow$ 2)
                \\4  & p &(Eliminacion $\wedge$ 3)
                \\5  & q  &(MP 4 2)
                \\6  & $q \wedge r $  &(Introduccion $\wedge$ 4 5)
                \\& $\vDash_N$
            \end{tabular}
            
            %ejercicio3.b
            \item \textbf{$\vdash_N (p \Rightarrow q) \Rightarrow (p \wedge r \Rightarrow q \wedge r)$}
            \\\begin{tabular}{l l r}
                   1  & $\vdash_N (p \Rightarrow q) \Rightarrow ((p \wedge r) \Rightarrow (q \wedge r)) $ & (premisa)
                \\ 2  & $(p \Rightarrow q) \vdash_N ((p \wedge r) \Rightarrow (q \wedge r)) $& (Eliminacion $\rightarrow$ 1)
                \\3  & $ (p \wedge r ) \vdash (q \wedge r) $ &(Eliminacion $\rightarrow$ 2)
                \\4  & p &(Eliminacion $\wedge$ 3)
                \\5  & q  &(MP 4 2)
                \\6  & $q \wedge r $  &(Introduccion $\wedge$ 4 5)
                \\& $\vDash_N$
            \end{tabular}
            
            %ejercicio3.c
            \item \textbf{$\vdash_N (p \Rightarrow q) \Rightarrow (p \vee r \Rightarrow q \vee r)$}
            \\\begin{tabular}{l l r}
                   1  & $\vdash_N (p \Rightarrow q) \Rightarrow ((p \vee r) \Rightarrow (q \vee r)) $ & (premisa)
                \\ 2  & $(p \Rightarrow q) \vdash_N ((p \vee r) \Rightarrow (q \vee r)) $& (Eliminacion $\rightarrow$ 1)
                \\3  & $ (p \vee r ) \vdash (q \vee r) $ &(Eliminacion $\rightarrow$ 2)
                \\4  & si p &(Eliminacion $\vee$ 3)
                \\5  & q  &(MP 4 2)
                \\6  & $q \vee r $  &(Introduccion $\vee$ 5)
                \\7  & si r &(Eliminacion $\vee$ 3)
                \\8  & $q \vee r $  &(Introduccion $\vee$ 7)
                \\& $\vDash_N$
            \end{tabular}
            
            %ejercicio3.d
            \item \textbf{$p \Rightarrow q, q \Rightarrow r \vee s, \neg s, p \vdash_N  r.$}
            \\\begin{tabular}{l l r}
                   1  & $p \Rightarrow q, q \Rightarrow r \vee s, \neg s, p \vdash_N  r$ & (premisa)
                \\ 2  &$p \Rightarrow q $ & (premisa)
                \\ 3  & $q \Rightarrow (r \vee s)$ & (premisa)
                \\ 4  & $\neg s $ & (premisa)
                \\ 5  & $ p $ & (premisa)
                \\ 6  & q & (MP 5 2)
                \\ 7  & $r \vee s $& (MP 6 3)
                \\ 8  & r& (Eliminacion $\vee$ 4 7 )
                \\& $\vDash_N$
            \end{tabular}
        \end{enumerate}
        
        %Ejercicio 4%
        \item \textbf{ . Demuestra que las reglas $E \wedge, F y I \Rightarrow  $son correctas }
        \begin{enumerate}
            \item \textbf{ E $ \wedge $}
            \\Sea $p\wedge q \vdash p$ 
            \\como es una premisa es verdadero, como la tabla de verdad de $\wedge$ es solo verdadero cuando ambos son verdaderos en necesario que p sea verdadero al igual que q.
            \\ \begin{tabular}{|c|c|c|}
                \hline
                p & q & $\wedge$ 
                \\ \hline
                0 & 0 & 0
                \\\hline
                0 & 1 & 0
                \\\hline
                1 & 0 & 0
                \\\hline
                1 & 1 & 1
                \\\hline
            \end{tabular}\\
            
            \item \textbf{F}
            \\Como el demostrar algo de esta forma es como una implicacion es decir de antecedentes  demostraremos unos consecuentes , si los antecedentes son falsos como en la tabla de verdad de la implicacion sera cierto cualquier consecuente.
            \\ \begin{tabular}{|c|c|c|}
                \hline
                p & q & $\Rightarrow$ 
                \\ \hline
                0 & 0 & 1
                \\\hline
                0 & 1 & 1
                \\\hline
                1 & 0 & 0
                \\\hline
                1 & 1 & 1
                \\\hline
            \end{tabular}\\
            
            \textbf{I $\rightarrow$}
            \\Si partimos de una premisa verdadera y concluimos algo  verdadero esto no es mas que otra forma de ver una implicacion
        \end{enumerate}
            
    \end{enumerate}
\end{document}